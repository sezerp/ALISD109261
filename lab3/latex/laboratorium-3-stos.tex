\documentclass{article}

\usepackage[cp1250]{inputenc}
\usepackage{polski}
\usepackage{algorithmic}
\usepackage{listings}
\usepackage{hyperref}

\mathchardef\hyphenmathcode=\mathcode`\-

\lstset{language=python, literate={-}{-}1}

\let\origlstlisting=\lstlisting
\let\endoriglstlisting=\endlstlisting
\renewenvironment{lstlisting}
    {\mathcode`\-=\hyphenmathcode
     \everymath{}\mathsurround=0pt\origlstlisting}
    {\endoriglstlisting}

% Default fixed font does not support bold face
\DeclareFixedFont{\ttb}{T1}{txtt}{bx}{n}{12} % for bold
\DeclareFixedFont{\ttm}{T1}{txtt}{m}{n}{12}  % for normal

% Custom colors
\usepackage{color}
\definecolor{deepblue}{rgb}{0,0,0.5}
\definecolor{deepred}{rgb}{0.6,0,0}
\definecolor{deepgreen}{rgb}{0,0.5,0}

\usepackage{listings}

% Python style for highlighting
\newcommand\pythonstyle{\lstset{
language=Python,
basicstyle=\ttm,
otherkeywords={self},             % Add keywords here
keywordstyle=\ttb\color{deepblue},
emph={MyClass,__init__},          % Custom highlighting
emphstyle=\ttb\color{deepred},    % Custom highlighting style
stringstyle=\color{deepgreen},
frame=tb,                         % Any extra options here
showstringspaces=false            % 
}}


% Python environment
\lstnewenvironment{python}[1][]
{
\pythonstyle
\lstset{#1}
}
{}

% Python for external files
\newcommand\pythonexternal[2][]{{
\pythonstyle
\lstinputlisting[#1]{#2}}}

% Python for inline
\newcommand\pythoninline[1]{{\pythonstyle\lstinline!#1!}}


\title{Algorytmy i struktury danych laboratorium 3\\Stos metody push i pop}

\author{Pawe� Zabczy�ski}
\date{2020\\ Czerwiec}

\begin{document}

\maketitle

\section*{Definicje}

Definicja elementu listy
\begin{python}
class Node(Generic[T]):
    def __init__(self, value: T):
        self.next: Node[T] = None
        self.previous: Node[T] = None
        self.value: T = value
\end{python}

\begin{python}
class List(Generic[T]):
    def __init__(self):
        self.head: Node[T] = None
        self.tail: Node[T] = None
        self.size: int = 0
        
    def is_empty(self):
        return self.head is None
\end{python}

\begin{python}
class ComparableNode(Generic[T]):
    def __init__(self, value: T):
        self.next: ComparableNode[T] = None
        self.previous: ComparableNode[T] = None
        self.value: T = value

    def ==(self, other: ComparableNode[T]) -> bool:
        if other is None:
            return False
        return self.value == other.value

    def !=(self, other):
        if other is None:
            return True
        return self.value != other.value

    def >(self, other: ComparableNode[T]'):
        return self.value > other.value

    def >=(self, other: ComparableNode[T]):
        return self.value >= other.value

    def <(self, other: ComparableNode[T]):
        return self.value < other.value

    def <=(self, other: ComparableNode[T]):
        return self.value <= other.value
\end{python}

\begin{python}
class SortedList(Generic[T]):
    def __init__(self):
        self.head: ComparableNode[T] = None
        self.tail: ComparableNode[T] = None
        self.size: int = 0
        
    def is_empty(self) -> bool:
        return self.head is None
\end{python}


\section*{Metody push}


Metoda dodaje element na pocz�tek listy

\begin{python}[caption=push\_front() lista jednokierunkowa]
def push_front(xs, e: T):
    node = Node[T](e)
    xs.size += 1
    if xs.head is None:
        xs.head = node
        xs.tail = node
    else:
        node.next = xs.head
        xs.head = node
\end{python}



\begin{python}[caption=push\_front() lista dwukierunkowa]
def push_front(xs: List[T], e: T):
    node = Node[T](e)
    xs.size += 1
    if xs.head is None:
        xs.head = node
        xs.tail = node
    else:
        xs.head.previous = node
        node.next = xs.head
        xs.head = node
\end{python}


\begin{python}[caption=push\_back() lista jednokierunkowa]
def push_back(xs: List[T], e: T):
    node = Node[T](e)
    xs.size += 1
    if xs.head is None:
        xs.head = node
        xs.tail = node
    else:
        xs.tail.next = node
        xs.tail = node
\end{python}


\begin{python}[caption=push\_back() lista dwukierunkowa]
def push_back(xs: List[T], e: T):
    node = Node[T](e)
    xs.size += 1
    if xs.head is None:
        xs.head = node
        xs.tail = node
    else:
        node.previous = xs.tail
        xs.tail.next = node
        xs.tail = node
\end{python}



\section*{Metody pop}

Metody pop - usuwaj�ce element list
\lstset{basicstyle=\ttfamily}
\begin{python}[caption=pop\_front() lista jednokierunkowa]
def pop_front(xs: List[T]):
    node = None
    if xs.is_empty():
        raise IndexError
    elif xs.size == 1:
        xs.size -= 1
        node = xs.head
        xs.head = None
        xs.tail = None
    else:
        xs.size -= 1
        node = xs.head
        xs.head = xs.head.next
        
    return node.value
\end{python}


\begin{python}[caption=pop\_front() lista dwukierunkowa]
def pop_front(xs: List[T]):
    node = None
    if xs.is_empty():
        raise IndexError
    elif xs.size == 1:
        xs.size -= 1
        node = xs.head
        xs.head = None
        xs.tail = None
    else:
        xs.size -= 1
        node = xs.head
        xs.head = xs.head.next
        xs.head.previous = None

    return node.value
\end{python}


\begin{python}[caption=pop\_back() lista jednokierunkowa]
def pop_back(xs: List[T]):
    node_result = None
    if xs.is_empty():
        raise IndexError
    elif xs.size == 1:
        xs.size -= 1
        node_result = xs.head
        xs.head = None
        xs.tail = None
    else:
        node = xs.head
        while node.next != xs.tail:
            node = node.next
            
        node_result = node.next
        xs.tail = node
        xs.tail.next = None
        xs.size -= 1
        
        return node_result.value
\end{python}


\begin{python}[caption=pop\_back() lista dwukierunkowa]
def pop_back(xs: List[T]):
    node = None
    if xs.is_empty():
        raise IndexError
    elif xs.size == 1:
        node = xs.head
        xs.head = None
        xs.tail = None
    else:
        node = xs.tail
        xs.tail = node.previous
        xs.tail.next = None
    xs.size -= 1
    
    return node.value
\end{python}


\begin{python}[caption=pop\_1() lista jednokierunkowa]
    def pop_1()(xs: List[T], e: T):
        if xs.is_empty():
            raise IndexError

        result = None
        node = xs.head
        if xs.head.value == e:
            result = xs.head
            xs.head = xs.head.next
            xs.size -= 1
        else:
            while node.next and node.next.value != e:
                node = node.next

            if node.next.value == e:
                result = node.next
                xs.size -= 1
                if node.next.next is not None:
                    node.next = node.next.next
                else:
                    xs.tail = node
                    node.next = None
            else:
                result = None

        return result.value if result else None
\end{python}

\begin{python}[caption=pop\_2() lista dwukierunkowa]
    def pop_2(xs: List[T], e: T):

        if xs.is_empty():
            raise IndexError

        result = None
        node = xs.head
        if node.value == e:
            result = xs.head
            xs.head = None
            xs.tail = None
            xs.size -= 1
        else:
            while node.next and node.next.value != e:
                node = node.next

            if node.next.value == e:
                result = node.next
                xs.size -= 1
                if node.next.next is not None:
                    node.next = node.next.next
                    node.next.previous = node
                else:
                    xs.tail = node
                    node.next = None
            else:
                result = None
\end{python}


\section*{Lista sortowane insert, delete}

\begin{python}[caption=insert() lista jednokierunkowa]
    def insert(xs: SortedList[T], e: T):
        new_node = ComparableNode[T](e)
        xs.size += 1
        if xs.is_empty():
            xs.head = new_node
            xs.tail = new_node
        elif xs.head > new_node:
            new_node.next = xs.head
            xs.head = new_node
        else:
            node = xs.head
            while node.next and node.next < new_node:
                node = node.next

            if node.next:
                new_node.next = node.next
                node.next = new_node
            else:
                node.next = new_node
                xs.tail = new_node
\end{python}

\begin{python}[caption=delete() lista jednokierunkowa]
 def delete(xs: SortedList[T], e: T) -> None:
        if xs.is_empty():
            raise IndexError

        result = None
        node = xs.head
        if xs.head.value == e:
            result = xs.head
            xs.head = xs.head.next
            xs.size -= 1
            if xs.size == 0:
                xs.tail = None
        else:
            while node.next and node.next.value != e:
                node = node.next

            if node.next.value == e:
                result = node.next
                xs.size -= 1
                if node.next.next is not None:
                    node.next = node.next.next
                else:
                    xs.tail = node
                    node.next = None
            else:
                result = None
\end{python}

\begin{python}[caption=insert() lista dwukierunkowe]
    def insert(xs: SortedList[T], e: T):
        new_node = ComparableNode[T](e)
        xs.size += 1
        if xs.is_empty():
            xs.head = new_node
            xs.tail = new_node
        elif xs.head > new_node:
            xs.head.previous = new_node
            new_node.next = xs.head
            xs.head = new_node
        else:
            node = xs.head
            while node.next and node < new_node:
                node = node.next

            if node.next or node > new_node:
                new_node.previous = node.previous
                new_node.next = node
                node.previous.next = new_node
                node.previous = new_node
            else:
                node.next = new_node
                new_node.previous = node
                xs.tail = new_node
\end{python}


\begin{python}[caption=delete() lista dwukierunkowa]
\end{python}


\section*{Podsumowanie}


Przykadowa implementacja list dwukierunkowej wraz z jednokierunkow� \href{https://github.com/sezerp/ALISD109261/tree/feature/lab3/lab3/linear_data_structures}{GitHub}




\end{document}
