\documentclass{article}

\usepackage[cp1250]{inputenc}
\usepackage{polski}
\usepackage{algorithmic}
\usepackage{listings}



\title{Algorytmy i struktury danych laboratorium 3\\Stos metody push i pop}

\author{Pawe� Zabczy�ski}
\date{2020\\ Czerwiec}

\begin{document}

\maketitle


\section*{Metody push}


Metoda  dodaje element na pocz�tek listy 


\lstset{language=python}
\begin{lstlisting}[caption=push\_front()]
def push_front(list: List):
	pass
\end{lstlisting}


Metoda push\_back \- dodaje element na koniec listy



\begin{lstlisting}[caption=push\_back()]
def push_back(list: List):
	pass
\end{lstlisting}


\section*{Metody pop}

\lstset{language=python}

\begin{lstlisting}[caption=pop\_back()]
def pop_back(list: List):
	pass
\end{lstlisting}



\begin{lstlisting}[caption=pop\_front()]
def pop_front(list: List):
	pass
\end{lstlisting}


\end{document}